\documentclass[11pt,a4paper]{article}
\usepackage[utf8]{inputenc}
\usepackage[T1]{fontenc}
\usepackage{lmodern}
\usepackage{geometry}
\geometry{margin=1in}
\usepackage{setspace}
\usepackage{hyperref}
\usepackage{titlesec}
\usepackage{abstract}

% Line spacing
\onehalfspacing

% Title formatting
\titleformat{\section}{\normalfont\large\bfseries}{\thesection.}{0.5em}{}
\titleformat{\subsection}{\normalfont\normalsize\bfseries}{\thesubsection}{0.5em}{}

% Title
\title{\textbf{Palladium: Ein Elektronisches Peer-to-Peer-Bezahlsystem}}
\author{Ursprünglich gestartet von NotRin7 (März 2024) \\\\ Whitepaper überarbeitet von davide3011, Maurito83 (Oktober 2025)}
\date{}

\begin{document}
\maketitle

\begin{abstract}
Palladium ist eine dezentralisierte digitale Währung, die sofortige, kostengünstige Transaktionen zwischen Parteien ermöglicht, ohne dass Zwischenhändler oder zentrale Autoritäten erforderlich sind. Abgeleitet vom Bitcoin-Protokoll, zielt Palladium darauf ab, Skalierbarkeits- und Effizienzprobleme zu lösen und gleichzeitig die Grundprinzipien der Dezentralisierung und Sicherheit zu wahren.
\end{abstract}

\section{Einleitung}
Palladium übernimmt die bahnbrechenden Konzepte, die von Bitcoin eingeführt wurden, und verfeinert gleichzeitig sein Protokoll, um den Transaktionsdurchsatz und die Nachhaltigkeit des Netzwerks zu optimieren. Durch eine Reihe von Verbesserungen und Anpassungen ist Palladium bestrebt, ein robustes und zugängliches digitales Bezahlsystem für eine breite Palette von Nutzern weltweit anzubieten.

\section{Übersicht}
Palladium funktioniert in einem Peer-to-Peer-Netzwerk und verwendet die Blockchain-Technologie zur Aufzeichnung und Überprüfung von Transaktionen. Im Gegensatz zu herkömmlichen Zahlungssystemen sind Palladium-Transaktionen unumkehrbar, transparent und zensurresistent. Durch den Verzicht auf Zwischenhändler gibt Palladium den Nutzern die direkte Kontrolle über ihre Gelder und Finanztransaktionen.

\section{Wichtige Merkmale}
\begin{enumerate}
    \item \textbf{Dezentralisierung} \\ 
    Palladium läuft auf einem dezentralen Netzwerk von Knoten (Nodes), was sicherstellt, dass es keinen zentralen Ausfallpunkt (Single Point of Failure) gibt, und verhindert Zensur oder Kontrolle durch eine zentrale Instanz.

    \item \textbf{Proof of Work (PoW)} \\ 
    Palladium verwendet den \textbf{SHA256-PoW-Algorithmus}, denselben Mining-Algorithmus, den auch Bitcoin verwendet. Das bedeutet, dass Palladium mit derselben Ausrüstung (ASIC-Minern) geschürft werden kann, was die Zugänglichkeit zur bestehenden Mining-Infrastruktur gewährleistet. PoW bietet robuste Sicherheit gegen böswillige Akteure und schafft gleichzeitig Anreize für Miner, das Netzwerk durch Blockvalidierung und -erstellung zu sichern.

    \item \textbf{Skalierbarkeit} \\ 
    Palladium ist so konzipiert, dass es vollständig skalierbar, Open Source und komplett dezentralisiert ist. Es werden Techniken implementiert, um den Transaktionsdurchsatz zu erhöhen und die Bestätigungszeiten zu verkürzen, was eine reibungslosere Benutzererfahrung und eine verbesserte Netzwerkeffizienz ermöglicht.

    \item \textbf{Kein Premine und faire Verteilung} \\ 
    Palladium startete \textbf{ohne Premine}, was eine faire und transparente Verteilung ab dem Genesis-Block gewährleistet. Die \textbf{anfängliche Blockbelohnung ist auf 50 PLM festgelegt}, mit einer \textbf{Halbierung alle 210.000 Blöcke}, was das Angebot reguliert und die Knappheit im Laufe der Zeit aufrechterhält.

    \item \textbf{Coinbase-Reife} \\ 
    Belohnungen aus dem Mining erfordern eine \textbf{Reife von 120 Blöcken}, bevor sie ausgegeben werden können, was die Sicherheit und Stabilität des Netzwerks gegen Ketten-Reorganisationen stärkt.

    \item \textbf{Schnelle Transaktionen} \\ 
    Im Gegensatz zu Bitcoins 10-minütigem Blockintervall erreicht Palladium eine \textbf{durchschnittliche Blockzeit von 2 Minuten}, was eine viel schnellere Transaktionsbestätigung ermöglicht und die Wartezeiten für Benutzer erheblich verkürzt.

    \item \textbf{Dynamische Schwierigkeitsanpassung} \\ 
    Palladium führt einen Mechanismus zur Schwierigkeitsanpassung \textbf{bei jedem Block} ein, der sicherstellt, dass das Netzwerk stabil und sicher bleibt, selbst wenn große Mengen an Hashrate in das System eintreten oder es verlassen. Diese Funktion verhindert Überlastungen und garantiert eine konsistente Blockgenerierung unabhängig von plötzlichen Änderungen der Mining-Leistung.
\end{enumerate}

\section{Anwendungsfälle}
Palladium kann für verschiedene Anwendungen genutzt werden, einschließlich, aber nicht beschränkt auf:
\begin{itemize}
    \item Zahlungen im Einzelhandel
    \item Überweisungen
    \item Mikrozahlungen
    \item Online-Einkäufe
    \item Grenzüberschreitende Transaktionen
\end{itemize}

\section{Fazit}
Palladium stellt eine bedeutende Evolution im Bereich der digitalen Währungen dar, die auf dem von Bitcoin gelegten Fundament aufbaut und gleichzeitig die zentralen Herausforderungen in Bezug auf Skalierbarkeit, Geschwindigkeit und Netzwerkanpassungsfähigkeit angeht. Durch die Bereitstellung eines dezentralisierten, sicheren und effizienten Zahlungssystems zielt Palladium darauf ab, die Art und Weise, wie Einzelpersonen und Unternehmen weltweit Finanztransaktionen durchführen, zu revolutionieren.

\end{document}