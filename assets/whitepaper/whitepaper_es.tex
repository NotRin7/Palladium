\documentclass[11pt,a4paper]{article}
\usepackage[utf8]{inputenc}
\usepackage[T1]{fontenc}
\usepackage{lmodern}
\usepackage{geometry}
\geometry{margin=1in}
\usepackage{setspace}
\usepackage{hyperref}
\usepackage{titlesec}
\usepackage{abstract}

% Line spacing
\onehalfspacing

% Title formatting
\titleformat{\section}{\normalfont\large\bfseries}{\thesection.}{0.5em}{}
\titleformat{\subsection}{\normalfont\normalsize\bfseries}{\thesubsection}{0.5em}{}

% Title
\title{\textbf{Palladium: Un sistema de efectivo electrónico entre pares}}
\author{Lanzado originalmente por NotRin7 (marzo de 2024) \\\\ Documento técnico (Whitepaper) revisado por davide3011, Maurito83 (octubre de 2025)}
\date{}

\begin{document}
\maketitle

\begin{abstract}
Palladium es una moneda digital descentralizada que permite transacciones instantáneas y de bajo coste entre partes, sin necesidad de intermediarios o autoridades centrales. Derivado del protocolo de Bitcoin, Palladium tiene como objetivo abordar los problemas de escalabilidad y eficiencia, preservando al mismo tiempo los principios básicos de descentralización y seguridad.
\end{abstract}

\section{Introducción}
Palladium hereda los conceptos pioneros introducidos por Bitcoin, al tiempo que refina su protocolo para optimizar el rendimiento de las transacciones y la sostenibilidad de la red. A través de una serie de mejoras y ajustes, Palladium se esfuerza por ofrecer un sistema de efectivo digital robusto y accesible para una amplia gama de usuarios en todo el mundo.

\section{Visión general}
Palladium opera en una red entre pares (peer-to-peer), utilizando la tecnología blockchain para registrar y verificar transacciones. A diferencia de los sistemas de pago tradicionales, las transacciones de Palladium son irreversibles, transparentes y resistentes a la censura. Al eliminar la dependencia de intermediarios, Palladium otorga a los usuarios control directo sobre sus fondos y transacciones financieras.

\section{Características clave}
\begin{enumerate}
    \item \textbf{Descentralización} \\ 
    Palladium opera en una red descentralizada de nodos, lo que garantiza que no haya un único punto de fallo y evita la censura o el control por parte de cualquier autoridad central.

    \item \textbf{Prueba de trabajo (Proof of Work - PoW)} \\ 
    Palladium utiliza el \textbf{algoritmo PoW SHA256}, el mismo algoritmo de minería utilizado por Bitcoin. Esto significa que Palladium puede minarse con el mismo equipo (mineros ASIC), garantizando la accesibilidad a la infraestructura minera existente. El PoW proporciona una seguridad robusta contra actores maliciosos, al tiempo que incentiva a los mineros a asegurar la red mediante la validación y creación de bloques.

    \item \textbf{Escalabilidad} \\ 
    Palladium está diseñado para ser totalmente escalable, de código abierto y completamente descentralizado. Se implementan técnicas para mejorar el rendimiento de las transacciones y reducir los tiempos de confirmación, lo que permite una experiencia de usuario más fluida y una mayor eficiencia de la red.

    \item \textbf{Sin preminado y distribución justa} \\ 
    Palladium se lanzó \textbf{sin preminado}, asegurando una distribución justa y transparente desde el bloque génesis. La \textbf{recompensa inicial por bloque se establece en 50 PLM}, con un \textbf{"halving" (reducción a la mitad) cada 210.000 bloques}, lo que regula el suministro y mantiene la escasez a lo largo del tiempo.

    \item \textbf{Madurez de la recompensa (Coinbase Maturity)} \\ 
    Las recompensas de la minería (coinbase) requieren una \textbf{madurez de 120 bloques} antes de poder ser gastadas, fortaleciendo la seguridad y estabilidad de la red contra reorganizaciones de la cadena.

    \item \textbf{Transacciones rápidas} \\ 
    A diferencia del intervalo de bloque de 10 minutos de Bitcoin, Palladium logra un \textbf{tiempo de bloque promedio de 2 minutos}, lo que permite una confirmación de transacciones mucho más rápida y reduce significativamente los tiempos de espera para los usuarios.

    \item \textbf{Ajuste dinámico de la dificultad} \\ 
    Palladium introduce un mecanismo de ajuste de dificultad \textbf{en cada bloque}, asegurando que la red permanezca estable y segura incluso cuando grandes cantidades de hashrate entran o salen del sistema. Esta característica previene la congestión y garantiza la generación de bloques de forma consistente, independientemente de los cambios repentinos en el poder de minería.
\end{enumerate}

\section{Casos de uso}
Palladium puede utilizarse para diversas aplicaciones, incluyendo, entre otras:
\begin{itemize}
    \item Pagos minoristas
    \item Remesas
    \item Micropagos
    \item Compras en línea
    \item Transacciones transfronterizas
\end{itemize}

\section{Conclusión}
Palladium representa una evolución significativa en el ámbito de las monedas digitales, construyendo sobre la base establecida por Bitcoin y abordando al mismo tiempo desafíos clave en escalabilidad, velocidad y adaptabilidad de la red. Al proporcionar un sistema de pago descentralizado, seguro y eficiente, Palladium tiene como objetivo revolucionar la forma en que individuos y empresas realizan transacciones financieras en todo el mundo.

\end{document}