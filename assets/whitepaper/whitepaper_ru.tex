\documentclass[11pt,a4paper]{article}
\usepackage[utf8]{inputenc}
\usepackage[T1]{fontenc}
\usepackage{lmodern}
\usepackage{geometry}
\geometry{margin=1in}
\usepackage{setspace}
\usepackage{hyperref}
\usepackage{titlesec}
\usepackage{abstract}

% Line spacing
\onehalfspacing

% Title formatting
\titleformat{\section}{\normalfont\large\bfseries}{\thesection.}{0.5em}{}
\titleformat{\subsection}{\normalfont\normalsize\bfseries}{\thesubsection}{0.5em}{}

% Title
\title{\textbf{Palladium: Одноранговая электронная денежная система}}
\author{Первоначально запущен NotRin7 (март 2024 г.) \\\\ Белая книга пересмотрена davide3011, Maurito83 (октябрь 2025 г.)}
\date{}

\begin{document}
\maketitle

\begin{abstract}
Palladium — это децентрализованная цифровая валюта, которая обеспечивает мгновенные и недорогие транзакции между сторонами без необходимости в посредниках или центральных органах власти. Основанный на протоколе Биткойна, Palladium стремится решить проблемы масштабируемости и эффективности, сохраняя при этом ключевые принципы децентрализации и безопасности.
\end{abstract}

\section{Введение}
Palladium наследует новаторские концепции, представленные Биткойном, одновременно совершенствуя свой протокол для оптимизации пропускной способности транзакций и устойчивости сети. Посредством ряда усовершенствований и корректировок Palladium стремится предложить надежную и доступную систему цифровых наличных денег для широкого круга пользователей по всему миру.

\section{Обзор}
Palladium работает в одноранговой сети, используя технологию блокчейн для записи и проверки транзакций. В отличие от традиционных платежных систем, транзакции Palladium необратимы, прозрачны и устойчивы к цензуре. Устраняя зависимость от посредников, Palladium предоставляет пользователям прямой контроль над своими средствами и финансовыми транзакциями.

\section{Ключевые особенности}
\begin{enumerate}
    \item \textbf{Децентрализация} \\ 
    Palladium работает в децентрализованной сети узлов, что исключает единую точку отказа и предотвращает цензуру или контроль со стороны какого-либо центрального органа.

    \item \textbf{Доказательство работы (Proof of Work - PoW)} \\ 
    Palladium использует \textbf{алгоритм SHA256 PoW} — тот же алгоритм майнинга, что и Биткойн. Это означает, что Palladium можно добывать на том же оборудовании (ASIC-майнерах), что обеспечивает доступ к существующей инфраструктуре майнинга. PoW обеспечивает надежную защиту от злоумышленников, одновременно стимулируя майнеров обеспечивать безопасность сети посредством проверки и создания блоков.

    \item \textbf{Масштабируемость} \\ 
    Palladium спроектирован как полностью масштабируемый, с открытым исходным кодом и полностью децентрализованный. Внедрены методы для повышения пропускной способности транзакций и сокращения времени их подтверждения, что обеспечивает более плавный пользовательский опыт и повышенную эффективность сети.

    \item \textbf{Отсутствие премайна и справедливое распределение} \\ 
    Palladium был запущен \textbf{без премайна}, что обеспечивает справедливое и прозрачное распределение с самого генезис-блока. \textbf{Начальная награда за блок установлена в 50 PLM}, с \textbf{уменьшением вдвое (халвингом) каждые 210 000 блоков}, что регулирует предложение и поддерживает дефицит с течением времени.

    \item \textbf{Зрелость Coinbase (награды за майнинг)} \\ 
    Награды за майнинг требуют \textbf{зрелости в 120 блоков}, прежде чем их можно будет потратить, что усиливает безопасность и стабильность сети против реорганизаций цепочки.

    \item \textbf{Быстрые транзакции} \\ 
    В отличие от 10-минутного интервала между блоками у Биткойна, Palladium достигает \textbf{среднего времени блока в 2 минуты}, что обеспечивает гораздо более быстрое подтверждение транзакций и значительно сокращает время ожидания для пользователей.

    \item \textbf{Динамическая корректировка сложности} \\ 
    Palladium внедряет механизм корректировки сложности \textbf{на каждом блоке}, гарантируя, что сеть остается стабильной и безопасной, даже когда большие объемы хэшрейта входят в систему или выходят из нее. Эта функция предотвращает перегрузки и гарантирует последовательную генерацию блоков независимо от внезапных изменений в мощности майнинга.
\end{enumerate}

\section{Сферы применения}
Palladium может использоваться для различных приложений, включая, но не ограничиваясь:
\begin{itemize}
    \item Розничные платежи
    \item Денежные переводы
    \item Микроплатежи
    \item Онлайн-покупки
    \item Трансграничные транзакции
\end{itemize}

\section{Заключение}
Palladium представляет собой значительную эволюцию в области цифровых валют, опираясь на фундамент, заложенный Биткойном, и одновременно решая ключевые проблемы масштабируемости, скорости и адаптируемости сети. Предоставляя децентрализованную, безопасную и эффективную платежную систему, Palladium стремится революционизировать способы, которыми частные лица и компании совершают финансовые транзакции по всему миру.

\end{document}