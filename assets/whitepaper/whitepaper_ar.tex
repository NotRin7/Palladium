\documentclass[11pt,a4paper]{article}
\usepackage[utf8]{inputenc}
\usepackage[T1]{fontenc}
\usepackage{lmodern}
\usepackage{geometry}
\geometry{margin=1in}
\usepackage{setspace}
\usepackage{hyperref}
\usepackage{titlesec}
\usepackage{abstract}

% Line spacing
\onehalfspacing

% Title formatting
\titleformat{\section}{\normalfont\large\bfseries}{\thesection.}{0.5em}{}
\titleformat{\subsection}{\normalfont\normalsize\bfseries}{\thesubsection}{0.5em}{}

% Title
\title{\textbf{بلاديوم: نظام نقد إلكتروني ند للند}}
\author{تم إطلاقه في الأصل بواسطة NotRin7 (مارس 2024) \\\\ تمت مراجعة الورقة البيضاء بواسطة davide3011, Maurito83 (أكتوبر 2025)}
\date{}

\begin{document}
\maketitle

\begin{abstract}
بلاديوم (Palladium) هي عملة رقمية لا مركزية تتيح إجراء معاملات فورية ومنخفضة التكلفة بين الأطراف، دون الحاجة إلى وسطاء أو سلطات مركزية. يهدف بلاديوم، المشتق من بروتوكول البيتكوين، إلى معالجة مخاوف قابلية التوسع والكفاءة مع الحفاظ على المبادئ الأساسية للامركزية والأمان.
\end{abstract}

\section{مقدمة}
يرث بلاديوم المفاهيم الرائدة التي قدمتها البيتكوين بينما يعمل على تحسين بروتوكوله لزيادة إنتاجية المعاملات واستدامة الشبكة. من خلال سلسلة من التحسينات والتعديلات، يسعى بلاديوم لتقديم نظام نقد رقمي قوي ومتاح لمجموعة واسعة من المستخدمين في جميع أنحاء العالم.

\section{نظرة عامة}
يعمل بلاديوم على شبكة ند للند (peer-to-peer)، مستخدمًا تقنية البلوكشين لتسجيل المعاملات والتحقق منها. على عكس أنظمة الدفع التقليدية، فإن معاملات بلاديوم غير قابلة للعكس، وشفافة، ومقاومة للرقابة. من خلال التخلص من الاعتماد على الوسطاء، يمنح بلاديوم المستخدمين تحكمًا مباشرًا في أموالهم ومعاملاتهم المالية.

\section{الميزات الرئيسية}
\begin{enumerate}
    \item \textbf{اللامركزية} \\ 
    يعمل بلاديوم على شبكة لا مركزية من العقد، مما يضمن عدم وجود نقطة فشل واحدة ويمنع الرقابة أو التحكم من قبل أي سلطة مركزية.

    \item \textbf{إثبات العمل (PoW)} \\ 
    يستخدم بلاديوم خوارزمية \textbf{SHA256 PoW}، وهي نفس خوارزمية التعدين التي تستخدمها البيتكوين. هذا يعني أنه يمكن تعدين بلاديوم بنفس المعدات (أجهزة ASIC)، مما يضمن إمكانية الوصول إلى البنية التحتية الحالية للتعدين. يوفر إثبات العمل أمانًا قويًا ضد الجهات الخبيثة بينما يحفز المعدنين لتأمين الشبكة من خلال التحقق من الكتل وإنشائها.

    \item \textbf{قابلية التوسع} \\ 
    تم تصميم بلاديوم ليكون قابلاً للتوسع بالكامل، ومفتوح المصدر، ولا مركزيًا تمامًا. يتم تطبيق تقنيات لتعزيز إنتاجية المعاملات وتقليل أوقات التأكيد، مما يتيح تجربة مستخدم أكثر سلاسة وكفاءة شبكة محسنة.

    \item \textbf{بدون تعدين مسبق وتوزيع عادل} \\ 
    تم إطلاق بلاديوم \textbf{بدون تعدين مسبق (no premine)}، مما يضمن توزيعًا عادلاً وشفافًا منذ الكتلة الأولى (genesis block). تم تحديد \textbf{مكافأة الكتلة الأولية عند 50 PLM}، مع \textbf{تنصيف (halving) كل 210,000 كتلة}، مما ينظم العرض ويحافظ على الندرة بمرور الوقت.

    \item \textbf{نضج مكافأة التعدين (Coinbase Maturity)} \\ 
    تتطلب المكافآت من التعدين \textbf{نضجًا يبلغ 120 كتلة} قبل أن يمكن إنفاقها، مما يعزز أمن الشبكة واستقرارها ضد عمليات إعادة تنظيم السلسلة.

    \item \textbf{معاملات سريعة} \\ 
    على عكس فاصل الكتل البالغ 10 دقائق في البيتكوين، يحقق بلاديوم \textbf{متوسط وقت كتلة يبلغ دقيقتين}، مما يتيح تأكيدًا أسرع للمعاملات ويقلل بشكل كبير من أوقات الانتظار للمستخدمين.

    \item \textbf{تعديل ديناميكي للصعوبة} \\ 
    يقدم بلاديوم آلية لتعديل الصعوبة \textbf{عند كل كتلة}، مما يضمن بقاء الشبكة مستقرة وآمنة حتى عند دخول كميات كبيرة من قوة التجزئة (hashrate) إلى النظام أو خروجها منه. تمنع هذه الميزة الازدحام وتضمن إنشاء كتل متسق بغض النظر عن التغييرات المفاجئة في قوة التعدين.
\end{enumerate}

\section{حالات الاستخدام}
يمكن استخدام بلاديوم لتطبيقات مختلفة، بما في ذلك على سبيل المثال لا الحصر:
\begin{itemize}
    \item مدفوعات التجزئة
    \item التحويلات المالية
    \item المدفوعات المصغرة
    \item المشتريات عبر الإنترنت
    \item المعاملات عبر الحدود
\end{itemize}

\section{الخاتمة}
يمثل بلاديوم تطورًا كبيرًا في مجال العملات الرقمية، بناءً على الأساس الذي وضعته البيتكوين مع معالجة التحديات الرئيسية في قابلية التوسع والسرعة وقدرة الشبكة على التكيف. من خلال توفير نظام دفع لا مركزي وآمن وفعال، يهدف بلاديوم إلى إحداث ثورة في طريقة مشاركة الأفراد والشركات في المعاملات المالية في جميع أنحاء العالم.
