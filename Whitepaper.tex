\documentclass[11pt,a4paper]{article}
\usepackage[utf8]{inputenc}
\usepackage[T1]{fontenc}
\usepackage{lmodern}
\usepackage{geometry}
\geometry{margin=1in}
\usepackage{setspace}
\usepackage{hyperref}
\usepackage{titlesec}
\usepackage{abstract}

% Line spacing
\onehalfspacing

% Title formatting
\titleformat{\section}{\normalfont\large\bfseries}{\thesection.}{0.5em}{}
\titleformat{\subsection}{\normalfont\normalsize\bfseries}{\thesubsection}{0.5em}{}

% Title
\title{\textbf{Palladium: A Peer-to-Peer Electronic Cash System}}
\author{Originally launched by NotRin7 (March 2024) \\\\ Whitepaper revised by davide3011, Maurito83 (October 2025)}
\date{}

\begin{document}
\maketitle

\begin{abstract}
Palladium is a decentralized digital currency that enables instant, low-cost transactions between parties, without the need for intermediaries or central authorities. Derived from the Bitcoin protocol, Palladium aims to address scalability and efficiency concerns while preserving the core principles of decentralization and security.
\end{abstract}

\section{Introduction}
Palladium inherits the pioneering concepts introduced by Bitcoin while refining its protocol to optimize transaction throughput and network sustainability. Through a series of enhancements and adjustments, Palladium strives to offer a robust and accessible digital cash system for a wide range of users worldwide.

\section{Overview}
Palladium operates on a peer-to-peer network, utilizing blockchain technology to record and verify transactions. Unlike traditional payment systems, Palladium transactions are irreversible, transparent, and resistant to censorship. By eliminating reliance on intermediaries, Palladium empowers users with direct control over their funds and financial transactions.

\section{Key Features}
\begin{enumerate}
    \item \textbf{Decentralization} \\ 
    Palladium operates on a decentralized network of nodes, ensuring no single point of failure and preventing censorship or control by any central authority.

    \item \textbf{Proof of Work (PoW)} \\ 
    Palladium utilizes the \textbf{SHA256 PoW algorithm}, the same mining algorithm used by Bitcoin. This means Palladium can be mined with the same equipment (ASIC miners), ensuring accessibility to existing mining infrastructure. PoW provides robust security against malicious actors while incentivizing miners to secure the network through block validation and creation.

    \item \textbf{Scalability} \\ 
    Palladium is designed to be fully scalable, open source, and completely decentralized. Techniques are implemented to enhance transaction throughput and reduce confirmation times, allowing for a smoother user experience and improved network efficiency.

    \item \textbf{No Premine and Fair Distribution} \\ 
    Palladium launched with \textbf{no premine}, ensuring fair and transparent distribution from the genesis block. The \textbf{initial block reward is set at 50 PLM}, with a \textbf{halving every 210,000 blocks}, regulating the supply and maintaining scarcity over time.

    \item \textbf{Coinbase Maturity} \\ 
    Rewards from mining require a \textbf{maturity of 120 blocks} before they can be spent, strengthening the network’s security and stability against chain reorganizations.

    \item \textbf{Fast Transactions} \\ 
    Unlike Bitcoin’s 10-minute block interval, Palladium achieves an \textbf{average block time of 2 minutes}, enabling much faster transaction confirmation and significantly reducing waiting times for users.

    \item \textbf{Dynamic Difficulty Adjustment} \\ 
    Palladium introduces a difficulty adjustment mechanism \textbf{at every block}, ensuring that the network remains stable and secure even when large amounts of hashrate enter or exit the system. This feature prevents congestion and guarantees consistent block generation regardless of sudden changes in mining power.
\end{enumerate}

\section{Use Cases}
Palladium can be utilized for various applications, including but not limited to:
\begin{itemize}
    \item Retail payments
    \item Remittances
    \item Micropayments
    \item Online purchases
    \item Cross-border transactions
\end{itemize}

\section{Conclusion}
Palladium represents a significant evolution in the realm of digital currencies, building upon the foundation laid by Bitcoin while addressing key challenges in scalability, speed, and network adaptability. By providing a decentralized, secure, and efficient payment system, Palladium aims to revolutionize the way individuals and businesses engage in financial transactions worldwide.

\end{document}
